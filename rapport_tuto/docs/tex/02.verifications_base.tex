\section{Ce qu'il faut faire en premier}

\subsection{Attendre}
Certains évènements extérieurs, tels que coupure de courant, foudre,
attaque informatique, etc. peuvent affecter les équipements réseaux à plus
ou moins large échelle. La résolution du problème n'est pas de votre
ressort, vous ne pouvez faire qu'attendre ! Dans ce genre de cas de
figure, le service technique diffuse à l'avance une note de service (si
l'évènement est plannifié, comme un changement de matériel), ou en informe
ses utilisateurs après coup, éventuellement par voie d'affichage si le
réseau est effectivement complètement inutilisable.\\
Noyer le service technique de requêtes ne fera pas plus avancer les choses.

\subsection{Vérifications de base}
Si votre accès au réseau est dégradé sans que vous n'ayez pour autant
connaissance d'un évènement perturbateur externe sur lequel vous n'avez
aucune prise, il se peut que vous soyez dans le cas de figure suivant :

\paragraph{Vous êtes débranché}
Si vous utilisez une connection filaire, vérifiez :

\begin{itemize}
\item que le câble est bien branché à votre ordinateur
\item que le câble est bien branché à la prise murale
\item que le câble est en bon état (pas à moitié entaillé par les roulettes
de votre chaise de bureau, par exemple)
\item que la prise murale est en bon état
\end{itemize}

\vskip 5pt

Si vous utilisez une connection sans fil, vérifiez :

\begin{itemize}
\item que la carte WiFi de votre ordinateur est bien activée (la plupart
des ordinateurs portable disposent d'un interrupteur physique pour la
liaison sans-fil, vérifiez que cet interrupteur est bien positionné sur ON)
\item que la borne WiFi sur laquelle vous vous connectez habituellement est
bien en marche (si vous savez où elle se trouve et qu'elle est accessible). 
\item que vous ne vous trouvez pas dans une cage de Faraday (connue pour
isoler des ondes électromagnétiques dont le WiFi fait partie).
\end{itemize}
