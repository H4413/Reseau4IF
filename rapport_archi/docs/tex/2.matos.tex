\section{Solution}

La problématique consiste à restructurer l'architecture d'un réseau
inter-établissement (réseau industriel et réseau de gestion) afin de
permettre de déplacer facilement des équipements d'un établissement
à un autre et de pouvoir limiter la pollution du réseau lors du broadcast
des variables globales des différents équipements industriels.

Dans une première ébauche de l'architecture, nous avions opté pour un
VLAN(Réseau Local Virtuel, regroupant un ensemble de machines de façon
logique et non physique) de niveau 3 afin de pouvoir broadcaster les variables
globales sur les différents VLAN des établissements. Cependant, compte tenu des
contraintes matérielles imposées par le cahier des charges, nous avons dû
modifier l'architecture étant donné que les équipements réseau imposés par le
client ne supportent pas le VLAN de niveau 3, mais seulement celui de niveau 2
(également appelé VLAN MAC, ce type de VLAN est beaucoup plus souple  que le
VLAN par port car le réseau est indépendant de la localisation de la station). 


Les avantages d'utiliser un VLAN:
plus de souplesse pour l'administration et les modifications du réseau car toute
l'architecture peut être modifiée par simple paramétrage des commutateurs
Gain en sécurité car les informations sont encapsulées dans un niveau
supplémentaire et éventuellement analysées
Réduction de la diffusion du trafic sur le réseau

Ces contraintes vont avoir un impact important sur l'organisation du réseau
puisque chaque VLAN sera alors isolé, le broadcast des variables globales vers
le réseau d'un autre établissement ne sera alors plus possible. 
Les utilisateurs pourront accéder à distance aux réseaux des établissements en
utilisant la fonction RAS (Remote Access Service) des routeurs qui permettent
d'établir une connexion sécurisée à  distance via le réseau téléphonique, en
fonction des droits attribués par le firewall. 

 \section{Nouvelle architecture matérielle}

Matériel à acquérir :

Afin de répondre à la problématique, les équipements réseaux suivant
ont été retenus:

\el
\begin{itemize}
\item[Routeur S@n 2000]
\begin{itemize}
\item[Rôle]: interconnexion de tous les équipements locaux et liaison VPN pour
    les équipements distants, pare-feu.
\item[Lieu]: un par site
\end{itemize}


	L'achat d'un routeur est discutable. En effet, les fonctionnalités
de routeur n'étaient pas indispensables à la mise en place d'un réseau
VPN (on aurait pu opter pour une solution de serveur VPN logiciel, sur
serveur dédié ou sur un serveur préexistant).
	Cependant, la matérialisation du réseau de l'AIPRAO est plus élégante.
En outre, la majorité des équipements sont situés au même emplacement que
ce routeur, et le fait que ces connexions ne soient pas cryptées et encapsulées
dans un tunnel VPN ne peut qu'améliorer les performances de l'ensemble.
	A fortiori, ce routeur représente un excellent compromis entre sécurité et
simplicité d'administration, étant données les conditions d'utilisation et de
maintenance du réseau de l'AIPRAO : faible criticité des données véhiculées,
exigeances de disponibilité modérées, compétences informatiques modérées de
l'administrateur habituel.

\el
\item[Switch administrable 8 ports TCS ESM 083F23F0]
\begin{itemize}
\item[Rôle] : Relier tous les équipements d'une même plateforme
\item[Débit] : 10 Base-T / 100 Base-TX
\end{itemize}
\el

\item[Switch administrable 16 ports TCS ESM 163F23F0]
\begin{itemize}
\item[Rôle] : Relier plusieurs équipements à l'intérieur d'une des
    trois salle de l'AIPRAO
\item[Débit] : 10 Base-T / 100 Base-TX
\item[Fonctionnalité]: Client FDR, SMTP V3, SNTP, filtrage multicast d’optimisation du
protocole Global Data, configuration par accès Web VLAN, IGMP Snooping,
RSTP (Rapide Scanning Tree Protocol), port prioritaire, controle des flux,
port sécurisé.
\end{itemize}
\el
\item[Switch administrables 22 ports TCS ESM 243F2CU0]
\begin{itemize}
\item[Rôle] : Relier tous les équipements du site central
\item[Débit] : 10 Base-T / 100 Base-TX
\end{itemize}

\end{itemize}
Eventuellement, il faudra également acquérir des cartes réseaux supplémentaires
pour les PC des plateformes, puisque chaque PC de plateforme devra être muni de
deux cartes réseau.



\section{Mise en place du VPN}

\begin{description}
\item	[Côté serveur], le la gestion du VPN est déléguée au routeur du site central.
La documentation de cette fonctionnalité du S@n2000 étant indisponible sur
internet, il convra de se renseigner auprès de Schneider lors de l'achat afin
de prendre connaissances des modalités de configuration.

\item	[Côté client], le script de démarrage sera modifié afin de se connecter
automatiquement au serveur VPN. Le client VPN devra être également fourni par
Schneider, puisqu'il n'existe pas de client VPN libre. L'appel au client VPN se
fera après l'établissement d'une connexion réseau sur le réseau local.

\end{description}
	Rappelons que les ordinateurs localisés dans le site principal se
connecterons directement au  réseau de l'AIPRAO, et ne devront donc pas lancer
ce client au démarrage. Si les administrateurs système souhaitent déployer un
unique script pour tout le parc, on pourra par exemple leur conseiller de tester
l'adresse IP de l'interface réseau dans leur script. Si cette IP appartient à la
plage réservée aux équipements de l'AIPRAO, cela signifie que l'ordinateur est
sur le site principal, et qu'il n'y a pas besoin de lancer le client VPN. Dans
le cas contraire, il faut lancer le  client VPN afin de se connecter au réseau
à distance.

	La mise en place d'un serveur d'authentification pour accéder aux service de
l'AIP est également indispensable. Sa mise en oeuvre sort toutefois du cadre de
cette étude,  comme spécifié dans le cahier des charges fonctionnel.
