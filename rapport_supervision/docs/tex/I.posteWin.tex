\vfill \pagebreak
\part{Monitoring depuis un poste Windows}

\subsection{Introduction}
Un serveur de supervision (SNMP/MRTG et/ou Nagios/NRPE) peut être configuré
puis utilisé sans problèmes depuis une plateforme Windows ; il suffit pour
cela de disposer d'un accès ssh vers la machine exécutant les logiciels de
supervision, puis d'utiliser son éditeur de texte préféré pour configurer à
volonté le serveur.\\

Pour interfacer Nagios et Windows, un programme tiers est nécessaire ; on
propose ici d'utiliser NSClient++ dans sa version 0.2.5e.

\subsection{Installation de NSClient++}

L'installation se fait à partir d'une invite de commande MS-Dos. Dans le
dossier où l'archive NSClient++ a été décompressée, lancer la commande
NSClient++ /install.

\subsection{Configuration de NSClient++}

Plusieurs choses doivent être faites :

\subsubsection{Permettre à NSClient++ d'interagir avec le bureau}

Il sera ainsi en mesure d'envoyer vers le serveur Nagios des informations
concernant l'ordinateur monitoré.
On procède de la façon suivante : Menu Démarrer/Accessoires/Outils
d'administration/Service ; une fois ici il faut trouver le service associé
à NSClient et lui donner l'autorisation.

\subsubsection{Configuration de NSClient++ proprement dite}

Elle s'effectue en éditant le fichier nsc.ini situé dans le répertoire
d'installation du client. On y

\begin{enumerate}
\item décommente les modules que l'on souhaite utiliser
\item redéfinit le mot de passe
\item ajout le serveur Nagios dans la section Allowed Hosts (les
différentes adresses sont ici séparées par des virgules)
\item définit le port associé à NSClient++ en décommentant la ligne
correspondante
\end{enumerate}

Après tout cela il faut relancer le service. On le fait grâce aux commandes
suivantes :

\begin{verbatim}
NSClient++ /stop
NSClient++ /start
\end{verbatim}

\subsubsection{Contrôles}

Pour s'assurer que l'installation fonctionne, il faut tout d'abord vérifier
que le port choisi à la section précédente est bien ouvert dans les
différents pare-feux protégeant l'ordinateur.\\
Une autre vérification est à mener au niveau du serveur Nagios. La commande
suivante :

\begin{verbatim}
cd /usr/local/nagios/libexec # a adapter suivant votre installation
./check_nt -H xxx.xxx.xxx.xxx -v CLIENTVERSION -p 12489
\end{verbatim}

\subsection{enregistrement du poste windows}

Il suffit pour cela d'éditer le fichier nagios.cfg en décommentant la ligne
correspondant à windows.cfg, puis d'éditer windows.cfg avec le nom et
l'adresse IP de la machine à monitorer. Ne pas oublier de redémarrer Nagios
pour que les changements soient pris en compte !
