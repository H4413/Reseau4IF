\section{Mise en place de procédures adaptées}

Le serveur DHCP est utilisé pour pourvoir des adresses IP aux machines et 
accessoires branchés localement. En vue d'obtenir un adresse pour un 
nouveau dispositif, il suffit de la connecter physiquement à une prise RJ45.
On veillera à configurer le DNS de manière manuel, méthode jugée la plus simple,
car les machines seront "statiques".

Voici les étapes pour ajouter une machine :
\begin{itemize}
\item[Configuration du DNS], en ajoutant un enregistrement de type A ou AAAA
sur le serveur DNS :
le\_nom\_de\_ma\_nouvelle\_machine IN A l\_adresse\_ip\_de\_ma\_machine
\item[Propogation des nouvelles informations], en modifiant l'enregistrement SOA
du serveur DNS, associée au sérial vu dans la partie précédente. On y modifie les
valeurs pour y mettre le jour d'aujourd'hui, sous la forme YYYYMMDD.
\end{itemize}
