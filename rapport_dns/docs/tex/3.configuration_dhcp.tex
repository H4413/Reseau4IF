\section{Configuration DHCP}

'x' correspond au site de 1 à 4 : 1 pour Lyon, 2 pour la salle GE, 3 pour Roanne,
4 pour Saint-Etienne.  


\subsection{Adresse réseau}

L'adresse du serveur DHCP sur chaque site sera 10.0.x.252.

\subsection{Masque de sous réseau}

255.255.255.0

\subsection{Plage d'adresses IP de réseau}

Les adresses IP des machines sont sur la plage 10.0.x.1 à 10.0.x.250.
On peut donc connecter 250 machines par sous-réseau.

\subsection{Durée du bail}

La durée du bail sera fixée à long pour éviter les conflits si trop de machines 
viennent à être connectées.

\subsection{Routeur}

L'adresse du routeur sur chaque site sera de 10.0.x.254.

\subsection{Adresse du DNS}

L'adresse du serveur DNS sur chaque site sera 10.0.x.253.

\subsection{Nom de l'étendue}

On va faire 2 étendues par site : de 10.0.x.1 à 10.0.x.64 pour les automates et
de 10.0.x.250 pour les autres machines.

\subsection{Plage de l'exclusion}

La plage de l'exclusion de chacune des 2 étendues est évidemment la plage 
de l'autre étendue.

\subsection{Associations MAC/IP}

Une adresse IP est attribuée à une adresse MAC tant qu'il reste des adresses 
disponibles pour les adresses MAC qui n'ont pas encore d'IP attribuée.
Si les adresses sont toutes déjà attribuées sur un site et qu'une nouvelle 
machine se connecte, l'adresse IP qui n'a pas été utilisée depuis le plus 
de temps est réattribuée à cette nouvelle adresse MAC.


