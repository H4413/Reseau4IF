\section{Proposition d'une architecture adaptée}

\subsection{Architecture multi-sites du service d'annuaire}

Répartition de l'annuaire LDAP sur plusieurs machines. On utilise la méthode de
réplication, on aura donc sur chaque site géographique une machine contenant 
une copie de l'annuaire. Cela permet de toujours se connecter au serveur le 
plus proche de l'utilisateur envoyant une requête et en plus une résistance 
aux pannes. En effet, si le serveur du site géographique est en panne, on peut
se connecter à un des autres serveurs sur site distant.

\subsection{Politique de nommage}

De manière générale, l'ensemble de numérotation comprenant des "xx", signifient 
que la numérotation peut aller de 01 à 99.

\subsubsection{Niveau Site}

Le nom du site sera choisi pour AIPRAO. La seconde extension sera choisie en fonction
du site ou se trouve l'objet réseau concerné. 

Exemple : ge.aiprao pour les objets du bâtiment du site GE. 

Nous choisissons pour l'ensemble des sites : 
\begin{itemize}
\item central.aiprao
\item ge.aiprao
\item roanne.aiprao
\end{itemize}

\subsubsection{Serveurs}

Nous créeons un sous domaine appelé serveur. Les serveurs auront pour nom leur 
principale fonctionnalité associée à leur numéro de serveur. 

Exemple : Nom du serveur LDAP situé au batiment central : ldap-01.serveur.central.aiprao

\subsubsection{Niveau Salle}

Les PC auront pour nom "pc-xx", xx étant le numéro du pc. Le premier PC aura 
pour numéro 01.

Exemple : le situé pc en GE aura pour nom pc-01.ge.aiprao .

\subsubsection{Site Central}

Les plateformes de manipulation auront pour nom de domaine "pl-xx".

Les automates auront pour nommage "a-xx". 

Exemple : le deuxième automate de la première plateforme aura pour nom : 

a-02.pl-01.central.aiprao

Les webcams auront pour nom "webcam-xx".

\subsection{Relation entre les serveurs}

Les différents serveurs communniquent entre eux en fonction du nom de domaine. 
Le DNS local permet d'identifier directement une machine sur son réseau local.


\subsection{Enregistrements}


