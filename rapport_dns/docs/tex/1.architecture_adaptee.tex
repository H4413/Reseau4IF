\section{Proposition d'une architecture adaptée}

\subsection{Architecture multi-sites du service d'annuaire}

Répartition de l'annuaire LDAP sur plusieurs machines. On utilise la méthode de
réplication, on aura donc sur chaque site géographique une machine contenant 
une copie de l'annuaire. Cela permet de toujours se connecter au serveur le 
plus proche de l'utilisateur envoyant une requête et en plus une résistance 
aux pannes. En effet, si le serveur du site géographique est en panne, on peut
se connecter à un des autres serveurs sur site distant.

\subsection{Architecture DNS}

On mettra en place un serveur DNS par site où se trouve l'AIP. Ceux-ci géreront 
les accès extérieurs au réseau au niveau de la connexion vers l'établissement. 
De plus ils connaîtront les autres serveurs DNS des sites distants de l'AIP. 

On veillera à ajouter les entrées de type (R) concernant les serveurs DNS de l'AIP sur les 
serveurs de niveaux supérieurs possédés par les établissements. 

\subsection{Politique de nommage}

De manière générale, l'ensemble de numérotation comprenant des "xx", signifient 
que la numérotation peut aller de 01 à 99.

\subsubsection{Niveau Site}

Le nom du site sera choisi pour AIPRAO. La seconde extension sera choisie en fonction
du site ou se trouve l'objet réseau concerné. 

Exemple : ge.aiprao pour les objets du bâtiment du site GE. 

Nous choisissons pour l'ensemble des sites : 
\begin{itemize}
\item central.aiprao
\item ge.aiprao
\item roanne.aiprao
\end{itemize}

\subsubsection{Serveurs}

Nous créons un sous domaine appelé serveur. Les serveurs auront pour nom leur 
principale fonctionnalité associée à leur numéro de serveur. 

Exemple : Nom du serveur LDAP situé au bâtiment central : ldap-01.serveur.central.aiprao

\subsubsection{Niveau Salle}

Les PC auront pour nom "pc-xx", xx étant le numéro du pc. Le premier PC aura 
pour numéro 01.

Exemple : le situé pc en GE aura pour nom pc-01.ge.aiprao .

\subsubsection{Site Central}

Les plateformes de manipulation auront pour nom de domaine "pl-xx".

Les automates auront pour nommage "a-xx". 

Exemple : le deuxième automate de la première plateforme aura pour nom : 

a-02.pl-01.central.aiprao

Les webcams auront pour nom "webcam-xx".

\subsection{Relation entre les serveurs}

Les différents serveurs communiquent entre eux en fonction du nom de domaine. 
Le DNS local permet d'identifier directement une machine sur son réseau local.

\subsection{Enregistrements}
\begin{itemize}
\item[A] nom d'hôte correspondant à une adresse IPv4;

[name] IN A [IPv4]
aiprao.insa-lyon.fr  IN      A       10.0.0.11

\item[AAAA] nom d'hôte correspond à une adresse IPv6;

[name] [number] IN AAAA [IPv6]
www.l.google.com.       77      IN      AAAA    2a00:1450:8004::68

\item[CNAME] record ou canonical name record qui permet de créer un alias entre 2 domaines;

[name] IN CNAME [name]
aiprao.insa-lyon.fr.              IN      CNAME       central.aiprao.insa-lyon.fr.

\item[PTR] record ou pointer record qui associe une adresse IP à un enregistrement de nom de domaine, aussi dit « reverse », il est le complémentaire de l'enregistrement A;

[IP] IN PTR [name]
10.0.0.11  IN   PTR     aiprao.insa-lyon.fr.

\item[NS] record ou name server record qui définit les serveurs DNS de ce domaine ;

[name] NS [name]
aiprao.insa-lyon.fr      NS  dns.aiprao.insa-lyon.fr.

\item[SOA] record ou Start Of Authority record qui donne les informations générales de la zone : serveur principal, courriel de contact, différentes durées dont celle d'expiration, numéro de série de la zone ;
[name] IN SOA [name] [email] [serial] [refresh] [retry] [expire] [minimumTTL]
aiprao.insa-lyon.fr	IN      SOA     dns.aiprao.insa-lyon.fr. admin.aiprao.insa-lyon.fr. 2010060311 43200 7200 1209600 3600

